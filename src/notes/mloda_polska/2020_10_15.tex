\subsubsection*{Fryderyk Nietzsche}
\begin{description}
    \item model życia
        \begin{itemize}
            \item jak Apollo
            \item jak Dionizos
                \begin{itemize}
                    \item dzieje się dużo, mnóstwo wydarzeń, na które człowiek nie ma wpływu
                \end{itemize}
        \end{itemize}
    \item jednostka genialna (wolna, doskonalsza od innych ludzi) osiąga pełną samoświadomość) --- człowiek może być niewolnikiem jedynie moralności, ograniczeń, reguł
    \item człowiek wychodzi poza granice dobra i~zła
    \item nowy typ bohatera literackiego
        \begin{itemize}
            \item wie, że od jego działań zależy to, jakim człowiekiem się stanie
            \item ,,każdy jest kowalem swojego losu''
        \end{itemize}
\end{description}
\subsubsection*{Leopold Staff \textit{Kowal}}
\begin{description}
    \item Sonet (4-4-3-3)
    \item Wypisz z~wiersza epitety i~podziel je na nacechowane
        \begin{itemize}
            \item dodatnio
                \begin{itemize}
                    \item kruszce drogocenne --- potencjał, który drzemie w~człowieku i~odpowiednio wykorzystany może uczynić go kimś wielkim
                    \item radosna otucha --- zdaje sobie sprawę, jak wielkie dzieło przed nim, zrobi wszystko, aby mu się udało
                    \item dzieło
                        \begin{itemize}
                            \item wielkie
                            \item pilne
                        \end{itemize}
                    \item serce
                        \begin{itemize}
                            \item hartowne
                            \item mężne
                            \item dumne
                            \item silne
                        \end{itemize}
                        Motor napędowy człowieka, centrum, pompa.
                    \item bezkształtna masa --- nie wiadomo co z~tego wyniknie, człowiek nie jest jeszcze ukształtowany, musi wykuć swoją silną, twardą osobowość
                    \item bezdenne otchłanie --- nie są przerażająca, pokazują ogrom potencjału
                \end{itemize}
            \item ujemnie
                \begin{itemize}
                    \item
                \end{itemize}
            \item neutralnie
                \begin{itemize}
                    \item kowadło
                        \begin{itemize}
                            \item twarde
                            \item stalowe
                        \end{itemize}
                \end{itemize}
        \end{itemize}
        Autor chce
        \begin{itemize}
            \item dowartościować
            \item umniejszyć
        \end{itemize}
    \item Jaki cel stawia sobie podmiot liryczny wiersza?
    \item Wskaż fragmenty, w~których można dostrzec inspirację filozofią Nietzschego.
    \item synekdocha
        \begin{itemize}
            \item \textit{pars pro toto} --- część zamiast całości --- symbol serca ma znaczenie ,,skondensowane''
            \item \textit{totum pro parte} --- całość zamiast części
        \end{itemize}
\end{description}
Co daje Staffowi ,,upchnięcie'' tego tekstu w~formę sonetu?


\subsubsection*{Impresjonistyczny charakter wiersza Leopolda Staffa pod tytułem \textit{Deszcz jesienny}}
\begin{description}
    \item piosenka (zwrotka --- refren)
    \item wiersz sylabotoniczny (ton --- akcent wyrazowy)
        \begin{itemize}
            \item monotonia
            \item jednostajność
        \end{itemize}
    \item instrumentacja głoskowa --- nagromadzenie podobnych głosek w~krótkim fragmencie tekstu --- melodyjność
    \item obraz poetycki (w~wierszu są cztery)
        \begin{enumerate}
            \item \textit{Wieczornych snów mary powiewne}
                \begin{itemize}
                    \item młodość i~smutek
                    \item rozpacz tak płacze
                \end{itemize}
            \item \textit{Wszak byłem na jakimś pogrzebie}
                \begin{itemize}
                    \item pożar spopielił zagrodę
                \end{itemize}
            \item \textit{Smutny szatan}
        \end{enumerate}
    \item Psychizacja krajobrazu (pejzażu)
\end{description}

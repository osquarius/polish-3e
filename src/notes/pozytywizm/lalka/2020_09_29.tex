\subsubsection*{Czy Wokulski był uczciwy?}
\begin{table}[h]
    \begin{tabularx}{\textwidth}{|X|X|X|c|}
        \hline
        \textbf{Wydarzenia i~postawy} & \textbf{Komentarz} & \textbf{Wartości} & \textbf{Uczciwość}\\
        \hline
        Praca w~winiarni Hopfera i~nauka & pomagał swojej rodzinie, choć ojciec tego nie rozumiał & praca, scjentyzm & \(+\)\\
        \hline
        Udział w~powstaniu styczniowym & idee romantyczne & patriotyzm & \(+\)\\
        \hline
        Małżeństwo z~Minclową & ożenił się, aby przeżyć, nie jest wyrachowany & oportunizm, pragmatyzm & \(+ / -\)\\
        \hline
        Pierwsze spotkanie z~Izabelą Łęcką --- narodziny miłości & Rzecki kazał mu się otrząsnąć z~żałoby po śmierci Minclowej & miłość, determinacja & \(+\)\\
        \hline
        Dostawy dla wojsk rosyjskich w~czasie wojny rosyjsko-tureckiej & patriota pozytywistyczny nigdy nie uznałby tego za uczciwe postępowanie & pieniądze & \(-\)\\
        \hline
        Pomoc materialna udzielona Łęckim & jawi mu się to jako jedyna droga do Izabeli, jest to tylko jego perspektywa, nie wyrachowanie & hojność, konsekwencja & \(+\)\\
        \hline
        Działalność charytatywna & jest to sposób życia --- wieży bardziej w~innych niż w~siebie; ufa, że każdy z~odpowiednią pomocą umie się odbić od dna & filantropia, praca u~podstaw, praca organiczna, szacunek dla człowieka & \(+\)\\
        \hline
        Historia z~aktorem Rossim & chce zaimponować Izabeli, krzywdzi tylko sam siebie, upokorzenie & konsekwencja w~działaniu, satysfakcja (z~zadowolenia panny Izabeli) & \(+\)\\
        \hline
        Wyjazd do Paryża & chce zakończyć swoją relację z~Łęckimi, oddać się nauce & rozum, miłość, niekonsekwencja, słabość w~stosunku do Izabeli, honor (nagrobek dla Stryja) & \(+ / -\)\\
        \hline
        Zaręczyny z~Izabelą & czuje, że to nie jest dobry pomysł, ale jest zaślepiony miłością, składa swoje życie na ręce Izabeli & miłość, szacunek, wierność & \(+ / -\)\\
        \hline
        Podróż pociągiem z~Izabelą i~Starskim & słyszy rozmowę, chce ostrzec parę, że rozumie angielski & zdecydowane działanie & \(+\)\\
        \hline
        Próba samobójcza & Wokulski uratowany przez Wysockiego, samobójstwo to w~świetle Dekalogu czyn całkowicie zły & konsekwencja & \(-\)\\
        \hline
        Tajemnicze zniknięcie & dba o~to, aby nikt pod względem majątkowym nie ucierpiał na jego zniknięciu & determinacja, odwaga & \(+\)\\
        \hline
    \end{tabularx}
\end{table}

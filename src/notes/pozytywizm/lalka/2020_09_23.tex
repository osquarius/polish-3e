
\subsubsection*{Nietypowy obraz arystokracji w~\textit{Lalce}}
\begin{description}
    \item Prezesowa Zasławska
        \begin{description}
            \item poznajemy ją, kiedy Wokulski przybywa na święcone na zaproszenie hrabiny
                \begin{itemize}
                    \item generacja romantyków (choć starsza od Rzeckiego)
                    \item łatwo się wzrusza, korzysta z~flakonika z~otrzeźwiącym płynem
                    \item przenosi sympatię do stryja Wokulskiego na samego Wokulskiego
                    \item chce sprawić stryjowi Wokulskiego nagrobek
                    \item ma poczucie zmarnowanego życia z~powodu nieszczęśliwej miłości
                    \item wie, że ,,błękitna krew'' nie musi nieść za sobą cnoty, uczciwości
                \end{itemize}
            \item w~Zasławku ukazana jako dobra gospodyni
                \begin{itemize}
                    \item pani na włościach
                    \item sprawnie organizuje pracę, sama pracuje tam, gdzie może
                    \item dba, aby jej gospodarstwo było w~dobrym stanie
                    \item szanuje każdego pracującego
                    \item Prusowi może chodzić o~to, żeby nie generalizować, uznając arystokratów za darmozjadów --- wśród tych osób są także ci, którzy liczą się z~innymi
                \end{itemize}
            \item wartości szanowane przez Zasławską
                \begin{itemize}
                    \item miłość romantyczna
                    \item drugi człowiek
                    \item uczucia innych
                    \item moralność, uczciwość
                    \item praca
                \end{itemize}
            \item dostrzega hipokryzję arystokracji
                \begin{itemize}
                    \item widzi, że szukają męża dla pieniędzy, a~trzymają z~kochankiem
                    \item uważa, że bogacze i~kochankowie to nie są prawdziwi ludzie
                    \item wysyła sygnały Wokulskiemu, żeby nie dał się ośmieszać Starskiemu, ostrzega go i~pokazuje konsekwencje związane z~miłością, również na własnym przykładzie
                \end{itemize}
        \end{description}
    \item Zasław
        \begin{description}
            \item Prus chce pokazać łączność z~przeszłością, że na prowincji też toczy się życie
        \end{description}
\end{description}

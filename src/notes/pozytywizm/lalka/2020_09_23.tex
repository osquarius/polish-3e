\subsubsection*{Studenci i~ich wizja świata}
\begin{description}
    \item kompletny chaos w~mieszkaniu
        \begin{itemize}
            \item nie szanują pokoju
            \item książki i~szachy --- rozrywki wskazujące na intelektualistów
        \end{itemize}
    \item podejście do życia
        \begin{itemize}
            \item sposób mówienia
                \begin{itemize}
                    \item prowokacyjny
                    \item pewność siebie
                    \item student nie zwraca się do Wirskiego grzecznościowo, np. ,,gadać'' zamiast ,,mówić''
                    \item chce ośmieszyć rozmówców pokrętną logiką, z~którą nie da się dyskutować
                \end{itemize}
            \item złamanie schematów obyczajowych
                \begin{itemize}
                    \item chodzenie nago
                \end{itemize}
            \item nieliczenie się z~konsekwencjami
                \begin{itemize}
                    \item nie płaci czynszu, bo nie uważa, że komuś się to należy
                    \item jest gotów zostać w~mieszkaniu dopóki go nie wyrzucą
                    \item czuje się uczciwie, oczekuje od społeczeństwa, że da mu możliwość zapłacenia czynszu
                \end{itemize}
            \item nauka --- scjentyzm
                \begin{itemize}
                    \item uczą się sami
                    \item nauczają innych
                    \item dążą do wyniesienia społeczeństwa na wyższy poziom cywilizacyjny
                \end{itemize}
        \end{itemize}
    \item sposób myślenia
        \begin{itemize}
            \item studenci uważają, że społeczeństwo ma tylko wymagania, a~nie daje możliwości do ich spełnienia
            \item mają nietypowe podejście do życia: cenią je, ponieważ sądzą, że jeżeli urodzili się w~społeczeństwie pełnym oczekiwań, to należą im się też warunki do spełnienia tych oczekiwań i~przyczynienia się do poprawy sytuacji. Z~drugiej strony, student nie przejmuje się niebezpieczeństwem związanym z~huśtaniem się w~oknie wysoko nad ziemią
            \item ich pretensje są uzasadnione w~stosunku do całego społeczeństwa, ale nie w~stosunku do konkretnego właściciela kamienicy. Student powinien uszanować to, co świat mu daje lub czego mu nie daje, podporządkować się realiom
            \item student tęskni za możliwością rozwoju, czymś, co pozwoli wszystkim uczciwie spełniać swoje obowiązki i~posiadać swoje przywileje
            \item Wirski traktuje studentów z~przymrużeniem oka, być może dlatego, że sam tęskni za lepszym światem (niekoniecznie socjalistyczną utopią, ale środowiskiem, które daje wszystkim możliwość godnego, uczciwego życia i~pracy; inni bohaterowie o~zapędach ,,zdrowo socjalistycznych'' to np. jeden z~subiektów, sądzi się tak także o~Wokulskim)
            \item studenci czują, że dają innym osobom zatrudnienie (pokój wynajmowało się niejako z~obsługą, np. praczkami, kucharkami, \ldots)
        \end{itemize}
    \item przyczyny patologii
        \begin{itemize}
            \item brak perspektyw, co budzi frustrację
            \item brak możliwości rozwoju intelektualnego studentów (podobnie jak Raskolnikow, choć na mniejszą skalę) popycha ich do działań niemoralnych, których w~normalnych okolicznościach nie popełniliby
            \item brudne miasto, dosłownie i~w~przenośni
            \item widzą dookoła siebie osoby dużo gorzej wykształcone, a~jednak znajdujące się w~lepszej sytuacji majątkowej i~posiadające bardziej znaczącą pozycję w~społeczeństwie
        \end{itemize}
\end{description}
\subsubsection*{Nietypowy obraz arystokracji w~\textit{Lalce}}
\begin{description}
    \item Prezesowa Zasławska
        \begin{description}
            \item poznajemy ją, kiedy Wokulski przybywa na święcone na zaproszenie hrabiny
                \begin{itemize}
                    \item generacja romantyków (choć starsza od Rzeckiego)
                    \item łatwo się wzrusza, korzysta z~flakonika z~otrzeźwiącym płynem
                    \item przenosi sympatię do stryja Wokulskiego na samego Wokulskiego
                    \item chce sprawić stryjowi Wokulskiego nagrobek
                    \item ma poczucie zmarnowanego życia z~powodu nieszczęśliwego miłościa
                    \item wie, że ,,błękitna krew'' nie musi nieść za sobą cnoty, uczciwości
                \end{itemize}
            \item w~Zasławku ukazana jako dobra gospodyni
                \begin{itemize}
                    \item pani na włościach
                    \item sprawnie organizuje pracę, sama pracuje tam, gdzie może
                    \item dba, aby jej gospodarstwo było w~dobrym stanie
                    \item szanuje każdego pracującego
                    \item Prusowi może chodzić o~to, żeby nie generalizować, uznając arystokratów za darmozjadów --- wśród tych osób są także Ci, którzy liczą się z~innymi
                \end{itemize}
            \item wartości szanowane przez Zasławską
                \begin{itemize}
                    \item miłość romantyczna
                    \item drugi człowiek
                    \item uczucia innych
                    \item moralność, uczciwość
                    \item praca
                \end{itemize}
            \item dostrzega hipokryzję arystokracji
                \begin{itemize}
                    \item widzi, że szukają męża dla pieniędzy, a~trzymają z~kochankiem
                    \item uważa, że bogacze i~kochankowie to nie są prawdziwi ludzie
                    \item wysyła sygnały Wokulskiemu, żeby nie dał się ośmieszać Starskiemu, ostrzega go i~pokazuje konsekwencje związane z~miłością, również na własnym przykładzie
                \end{itemize}
        \end{description}
    \item Zasław
        \begin{description}
            \item Prus chce pokazać łączność z~przeszłością, że na prowincji też toczy się życie
        \end{description}
\end{description}

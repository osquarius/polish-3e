\subsubsection*{Przedstawiciele arystokracji}
\begin{description}
    \item Krzeszowscy
        \begin{itemize}
            \item nie pobrali się do końca z~miłości, ale zależy im nawzajem na sobie (kłócą się), łączy ich jakieś uczucie
            \item baron chce rozwodu, a~baronowa nie, ponieważ ma duży kapitał i~nie chce dzielić majątku
            \item spotkało ich trudne doświadczenie --- śmierć córki --- różnią się na poziomie przeżywania tego wydarzenia:
                \begin{itemize}
                    \item baron próbuje utopić smutek w~rozrzutnym życiu
                    \item baronowa kontempluje to wydarzenie, prowadzi bardzo ostrożne i~oszczędne życie
                \end{itemize}
            \item przedmioty kłótni
                \begin{itemize}
                    \item rzeczy stosunkowo nieważne --- pieniądze
                    \item baronowa wykupuje weksle barona i~spłaca jego długi
                        \begin{itemize}
                            \item być może zależy jej, aby zatrzymać męża przy sobie
                            \item może zależeć jej na zachowaniu honoru nazwiska
                        \end{itemize}
                \end{itemize}
            \item jak poznajemy małżeństwo
                \begin{itemize}
                    \item w~sklepie Wokulskiego
                    \item baronowa jest skąpa, być może taka oszczędność nie przystaje osobie z~takim majątkiem (chociaż być może to dlatego ma ten majątek)
                    \item narrator nie jest obiektywny, zwraca uwagę na wygląd Krzeszowskiej, używa określeń typu ,,oryginalny jegomość'' (lekko pobłażliwych, sympatyzujących)
                    \item baron jest rozrzutny
                \end{itemize}
            \item małżeństwo przedstawione jest jako grupa dziwaków, którzy dla pragmatycznego Wokulskiego są zupełnie niezrozumiali
        \end{itemize}
    \item baron Dalski
        \begin{itemize}
            \item szukał miłości, był przekonany że znalazł ją z~Eweliną, nie chciał widzieć złych stron
            \item nie dostrzega oczywistego romansu Janockiej ze Starskim
            \item jest to postać trzecioplanowa, która dopełnia epickiego rozmachu w~ukazaniu panoramy społeczeństwa w~\textit{Lalce}
        \end{itemize}
    \item książę
        \begin{itemize}
            \item rzuca frazesami
            \item czuje się członkiem klasy wybranej, w~przeciwieństwie do zwyczajnego tłumu
            \item mieszka we własnym pałacu
            \item oderwany od rzeczywistości, ma romantyczną wizję narodu, ale nie podejmuje żadnych konkretnych działań (bo i~tak nikt ich od niego nie oczekuje)
            \item adwokat uważa arystokrację za zepsutą, zarażającą również zdrowe warstwy społeczne
        \end{itemize}
\end{description}